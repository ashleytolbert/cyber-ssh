\documentclass[11pt]{article}

\usepackage{hyperref, enumitem, times, microtype}
\usepackage[top=1in, bottom=1in, left=1in, right=1in]{geometry}

\newcommand{\slim}{\vspace{\baselineskip}}

\begin{document}

\noindent \textbf{Title:} Policy-friendly remote access to computer resources: the successor to SSH

\section*{Abstract}

\section{Participants}

\noindent \textbf{Ashley Tolbert} is a cybersecurity specialist at the SLAC
National Accelerator Laboratory. \ldots

\slim

\noindent \textbf{Erwin Lopez} is deputy chief information security officer at
SLAC National Accelerator Laboratory. \ldots

\slim

\noindent \textbf{David Mazi\`{e}res} is a professor of computer
science at Stanford University, where he leads the Secure Computer
Systems research group. He also serves as Chief Scientist at Stellar
Development Foundation, and is a founder of Intrinsic,
Inc. Prof.~Mazi\`{e}res's research interests include Operating Systems
and Distributed Systems, with a particular focus on security. Some of
the projects he and his students have worked on include SFS (a
self-certifying network file system), SUNDR (a file system that
introduced the notion of fork linearizability), Kademlia (a widely
used peer-to-peer routing algorithm), Coral (a peer-to-peer content
distribution network), HiStar (a secure operating system based on
decentralized information flow control), tcpcrypt (a TCP option
providing forward-secure encryption), Hails (a web framework that can
preserve privacy while incorporating untrusted third-party apps), Dune
(a driver granting linux processes safe access to privileged CPU
features), and COWL (an information-flow-control-based browser
security architecture). Prof.~Mazi\`{e}res has several awards
including an Oakland distinguished paper award (2014), Sloan award
(2002), USENIX best paper award (2001), NSF CAREER award (2001), MIT
Sprowls best thesis in computer science award (2000).

\slim

\noindent \textbf{Keith Winstein} is an assistant professor of computer science
and, by courtesy, of law. Winstein created the Mosh (mobile shell)
tool and the State Synchronization Protocol, which is estimated to
have more than a million users. He and his colleagues and students
developed the Sprout protocol for mobile networking, the Remy tool for
automated protocol design, the ExCamera system for massively parallel
video encoding on cloud microservices, and the Lepton system, which
has re-compressed hundreds of petabytes of image files. Winstein has
received the Usenix NSDI Community Award (2017), a Google Faculty
Research Award (2015 \& 2017), the ACM SIGCOMM Doctoral Dissertation
Award for Outstanding PhD Thesis in Computer Networking and Data
Communication (2015), a Sprowls award for best thesis in computer
science at MIT (2014), and the Applied Networking Research Prize
(2013). Winstein previously served as a staff reporter at The Wall
Street Journal (2007--2010) and consulting engineer at Dropbox
(2014--2017).

\slim

\noindent \textbf{Dima Kogan} is a doctoral student in computer science,
advised by Profs.~Winstein and Mazi\`{e}res. \ldots

\slim

\noindent \textbf{Riad S.~Wahby} is a doctoral student in computer science,
advised by Profs.~Winstein and Mazi\`{e}res. He was previously a
research scientist in the NYU Computer Science department and a
visiting researcher at the University of Texas at Austin. Before
entering computer-science research, Wahby spent ten years as a Staff
Design Engineer building analog and mixed-signal integrated circuits
at Silicon Labs.


\section{Proposed Research Agenda}

The government~\cite{cyberframework, nistSSH, trumpeo}.

\section{Timeline}

This proposal is for a one-year pilot project while the PIs work to
raise money to support a longer-term effort.

\section{Publication Plan}

\section{Budget}

We are asking for support for one Ph.D.~student (Greg Hill), plus
\$20,000 in equipment to build and deploy twenty measurement/antidote
boxes. CHAI has agreed to fund Greg Hill's travel to its sites in
Africa. Year 1 of the pilot project will cost \$97,825 out of gift
funds.

\slim

\noindent Annual budget:

\slim

\noindent \begin{tabular}{ll|l}
& \bf Amount & \bf \$ \\
Student & Academic-year 50\% RA & \$29,163 \\
        & Summer 90\% RA & \$19,442 \\

Tuition & Academic-year & \$19,094 \\
        & Summer        & \$1,909 \\

Benefits & Student & \$2,527 \\

Equipment &         & \$20,000 \\

\hline

Facility fee & 8\% & \$5,690 \\
Annual total & & \$97,825 \\

\end{tabular}

\slim

The Cyber Initiative previously funded part of the development of the
``self-incentivizing networks'' concept that is one of the mechanisms
we intend to deploy in the field (June 2015 funding round, \$30,000).

{\footnotesize

\bibliographystyle{abbrv}
\bibliography{cyber}
}

\end{document}
